\documentclass[12pt]{article}
\newcommand\tab[1][1cm]{\hspace*{#1}}
\usepackage[utf8]{inputenc}
\usepackage{listings}
\usepackage{hyperref}
\usepackage{color}
\pagenumbering{gobble}
\usepackage{changepage}

\usepackage{makecell}

\definecolor{codegreen}{rgb}{0,0.6,0}
\definecolor{codegray}{rgb}{0.5,0.5,0.5}
\definecolor{codepurple}{rgb}{0.58,0,0.82}
\definecolor{backcolour}{rgb}{0.95,0.95,0.92}

\hypersetup{
	colorlinks,
	citecolor=black,
	filecolor=black,
	linkcolor=black,
	urlcolor=black
}

\lstdefinestyle{mystyle}{
	backgroundcolor=\color{backcolour},   
	commentstyle=\color{codegreen},
	keywordstyle=\color{magenta},
	numberstyle=\tiny\color{codegray},
	stringstyle=\color{codepurple},
	basicstyle=\footnotesize,
	breakatwhitespace=false,         
	breaklines=true,                 
	captionpos=b,                    
	keepspaces=true,                 
	numbers=left,                    
	numbersep=5pt,                  
	showspaces=false,                
	showstringspaces=false,
	showtabs=false,                  
	tabsize=2
}

\lstset{style=mystyle}

\newcommand{\titledate}[2][2.5in]{%
	\noindent%
	\begin{tabular}{@{}p{#1}@{}}
		\\ \hline \\[-.75\normalbaselineskip]
		#2
	\end{tabular} \hspace{1in}
	\begin{tabular}{@{}p{#1}@{}}
		\\ \hline \\[-.75\normalbaselineskip]
		Date
	\end{tabular}
}

% for forcing tables to fit
\usepackage{changepage}

\begin{document}
	

\begin{titlepage}
	
\author{Josef Bostik\\
	Eric Pereira\\
	Ryan Wojtyla\\}
\date{November 26\textsuperscript{th}, 2018}
\title{Upgrade and Update of Computer Systems within Dr. Hohlmann’s High Energy Physics (HEP) Research Group Progress Evaluation: Milestone 3}

\maketitle

\end{titlepage}

\tableofcontents

\newpage \pagenumbering{arabic}

\section{HEP Senior Design}

\begin{itemize}
	\item Josef Bostik - \href{mailto:jbostik2015@my.fit.edu}{jbostik2015@my.fit.edu}
	\item Eric Pereira - \href{mailto:epereira2015@my.fit.edu}{epereira2015@my.fit.edu }
	\item Ryan Wojtyla - \href{mailto:rwojtyla2015@my.fit.edu}{rwojtyla2015@my.fit.edu}
\end{itemize}

\section{Faculty Sponsor}

\tab Eraldo Ribeiro - \href{mailto:eribeiro@fit.edu}{eribeiro@fit.edu}

\section{Client}

\tab Marcus Hohlmann - \href{mailto:hohlmann@fit.edu}{hohlmann@fit.edu} \\ 
\tab Head of the Florida Tech HEP group

\section{Meeting with Faculty Sponsor}

\section{Meeting with Client}

\begin{itemize}
\item[] 29 October 2019
\end{itemize}

\section{Progress of current Milestone}

\begin{adjustwidth}{-2cm}{}
  \begin{center}
    \begin{tabular}{|c|c|c|c|c|c|}
      \hline
      Task & \% Completion & Ryan & Eric & Josef & To Do \\
      \hline
      Repair Existing MTS & 70\% & 20\% & 40\% & 10\% & FEC firmware \\
      Prepare Development MTS Machine & 70\% & 20\% & 10\% & 40\% & repair
                                                                    software \\
      Boot Cluster into ROCKS & 80\% & 60\% & 10\% & 10\% & fix boot issues \\
      GEM Machines & 80\% & 10\% & 60\% & 10\% & continue reorganization \\
      \hline   
    \end{tabular}
  \end{center}
\end{adjustwidth}

\section{Discussion - Current Milestone}

\subsection{Existing MTS Progress}

\tab Part way through the month, one of the MTS's Front End Controllers (FECs)
failed to transmit any data; it was broken. Fortunately, we had a spare FEC that
we could put in its place. Unfortunately, however, this FEC was of a different
model, and it was not working properly with the rest of the hardware. After some
troubleshooting, it was determined that this newer FEC has a newer version of
firmware that is incompatible with the older version installed on the other
FECs. Since we do not have the firmware's binary laying around, we are asking
former graduate students and other researchers familiar with the MTS if they
happen to have a copy to send us. 

\subsection{Development MTS Computer Progress}

\tab While trying to use the required software we had just installed, namely
AMORE, it became apparent to us that there were some serious compatibility
issues. Turns out, the latest version of AMORE depends not on the latest version
of ROOT, ROOT 6, but the previous version, ROOT 5. After we had discovered this
issue, we began the task of uninstalling ROOT 6 and installing ROOT 5. The
process of installing ROOT 5 presented us with another issue: it was having a
hard time building. After some troubleshooting and research, we discovered that
we were simply attempting to build the software in the incorrect directory. With
this solved, ROOT 5 was built and it is installed. AMORE, however, is still
giving us trouble.

\subsection{Computing Cluster}

\tab A considerable amount of time was spend playing with the CentOS 7 images
discovered on the drive in the CE. It was discovered that, with the correct GRUB
configuration, the images can be loaded onto the machine. The images, however,
do not work; they kick the user straight to ``emergency mode''. The reason for
the boot failure is cited to be ``Failed to start Switch Root''. A meeting has
been scheduled with the IT department to explore possible causes of the original
internet connectivity issue with Anaconda, and perhaps some insight into this
new problem can be gleaned from the meeting as well.

\subsection{GEM Machines}


\section{Parts Worked On}

\subsection{Josef Bostik}

\begin{itemize}
\item Installing and building ROOT 5 on development MTS machine.
\item Troubleshooting AMORE on the development MTS machine.
\end{itemize}

\subsection{Eric Pereira}

\begin{itemize}
\item Testing two hard drives for GEM Computers
\item Fixing MySQL installation error on GEM computers
\item Created script and automated graphing data on Rapberry PI for GEM computers
\end{itemize}

\subsection{Ryan Wojtyla}

\begin{itemize}
\item Assisted with existing MTS troubleshooting.
\item Troubleshooting AMORE on the development MTS machine.
\item Attempted to boot into found images on the cluster.
\end{itemize}

\section{Task Matrix - Next Milestone}

\begin{center}
  \begin{tabular}{|c|c|c|c|}
    \hline
    Task & Ryan & Eric & Josef \\
    \hline
    Repair Existing MTS & 40\% & 30\% & 30\% \\
    Prepare Development MTS & 40\% & 20\% & 40\% \\
    Begin Wrapper Planning & 30\% & 20\% & 50\% \\
    Boot Cluster into ROCKS & 80\% & 10\% & 10\% \\
    Reorganize Highbay & 10\% & 20\% & 70\% \\
    \hline
  \end{tabular}
\end{center}

\section{Discussion - Next Milestone}

\subsection{Existing MTS}

\tab Once the new FEC is inundated with the correct version of firmware, data
collection in its present state may resume. The MTS, however, is stilled marred
by its prior issues of potentially faulty detectors and bugs with data
processing.

\subsection{Development MTS Computer}

\tab The development MTS machine still needs two critical components to be
completed before serious work on the wrapper application can begin: AMORE needs
to be made functional and drivers for the MTS hardware need to be
installed. While these two issues are being worked on, however, work can begin
on planning how the wrapper application will be constructed, at least at a high
level.

\subsection{Computing Cluster}

\tab With the discovered images found to be nonfunctional, we are running out of
options for what we are able to do. Aside from the scheduled meeting with the IT
department, we are left to investigate the images' ``Failed to start Switch
Root'' boot errors. If the original internet connectivity issue is solved during
the meeting, or if the boot error is resolved, we will finally be able to
install ROCKS onto the cluster.

\subsection{GEM Machines}


\section{Sponsor Feedback}

\subsection{Existing MTS}

\vspace{1in}

\subsection{Development MTS Machine}

\vspace{1in}

\subsection{Computing Cluster}

\vspace{1in}

\subsection{GEM Computers}

\vspace{1in}


\subsection{Sponsor Signature}
\vspace{.5in}
 \titledate{Sponsor Signature}
 
\section{Sponsor Evaluation}
\begin{tabular}{|c|c|c|c|c|c|c|c|c|c|c|c|c|c|c|c|c|}
	\hline
	Josef Bostik & 0 & 1 & 2 & 3 & 4 & 5 & 5.5 & 6 & 6.5 & 7 & 7.5 & 8 & 8.5 & 9 & 9.5 & 10\\
	\hline
	Eric Pereira & 0 & 1 & 2 & 3 & 4 & 5 & 5.5 & 6 & 6.5 & 7 & 7.5 & 8 & 8.5 & 9 & 9.5 & 10\\
	\hline
	Ryan Wojtyla & 0 & 1 & 2 & 3 & 4 & 5 & 5.5 & 6 & 6.5 & 7 & 7.5 & 8 & 8.5 & 9 & 9.5 & 10\\
	\hline
\end{tabular}

\end{document}
