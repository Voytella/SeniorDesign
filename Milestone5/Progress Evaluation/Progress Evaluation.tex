\documentclass[12pt]{article}
\newcommand\tab[1][1cm]{\hspace*{#1}}
\usepackage[utf8]{inputenc}
\usepackage{listings}
\usepackage{hyperref}
\usepackage{color}
\pagenumbering{gobble}
\usepackage{changepage}

\usepackage{makecell}

\definecolor{codegreen}{rgb}{0,0.6,0}
\definecolor{codegray}{rgb}{0.5,0.5,0.5}
\definecolor{codepurple}{rgb}{0.58,0,0.82}
\definecolor{backcolour}{rgb}{0.95,0.95,0.92}

\hypersetup{
	colorlinks,
	citecolor=black,
	filecolor=black,
	linkcolor=black,
	urlcolor=black
}

\lstdefinestyle{mystyle}{
	backgroundcolor=\color{backcolour},   
	commentstyle=\color{codegreen},
	keywordstyle=\color{magenta},
	numberstyle=\tiny\color{codegray},
	stringstyle=\color{codepurple},
	basicstyle=\footnotesize,
	breakatwhitespace=false,         
	breaklines=true,                 
	captionpos=b,                    
	keepspaces=true,                 
	numbers=left,                    
	numbersep=5pt,                  
	showspaces=false,                
	showstringspaces=false,
	showtabs=false,                  
	tabsize=2
}

\lstset{style=mystyle}

\newcommand{\titledate}[2][2.5in]{%
	\noindent%
	\begin{tabular}{@{}p{#1}@{}}
		\\ \hline \\[-.75\normalbaselineskip]
		#2
	\end{tabular} \hspace{1in}
	\begin{tabular}{@{}p{#1}@{}}
		\\ \hline \\[-.75\normalbaselineskip]
		Date
	\end{tabular}
}

% for forcing tables to fit
\usepackage{changepage}

\begin{document}
	

\begin{titlepage}
	
\author{Josef Bostik\\
	Eric Pereira\\
	Ryan Wojtyla\\}
\date{March 18\textsuperscript{th}, 2019}
\title{Revamp of High Energy Physics Laboratory's Computer Systems: Milestone 5}

\maketitle

\end{titlepage}

\tableofcontents

\newpage \pagenumbering{arabic}

\section{High Energy Physics (HEP) Senior Design}

\begin{itemize}
	\item Josef Bostik - \href{mailto:jbostik2015@my.fit.edu}{jbostik2015@my.fit.edu}
	\item Eric Pereira - \href{mailto:epereira2015@my.fit.edu}{epereira2015@my.fit.edu }
	\item Ryan Wojtyla - \href{mailto:rwojtyla2015@my.fit.edu}{rwojtyla2015@my.fit.edu}
\end{itemize}

\section{Faculty Sponsor}

\tab Eraldo Ribeiro - \href{mailto:eribeiro@fit.edu}{eribeiro@fit.edu}

\section{Client}

\tab Marcus Hohlmann - \href{mailto:hohlmann@fit.edu}{hohlmann@fit.edu} \\ 
\tab Head of the Florida Institute of Technology HEP group

\section{Meeting with Faculty Sponsor}

\begin{itemize}
	\item 18 March 2019
\end{itemize}
\section{Meeting with Client}

\begin{itemize}
	\item 11 February 2019
	\item 18 February 2019
	\item 25 February 2019
	\item 11 March 2019
\end{itemize}

\section{Progress of current Milestone}

\begin{adjustwidth}{-2.5cm}{}
  \begin{center}
      \begin{tabular}{|c|c|c|c|}
    	\hline
    	Task & Progress & Notes\\
    	\hline
    	Continue to Care for Existing MTS & 40\% & \\
    	Compile Instructions for MTS Operation & 50\% & improve upon provided
                                                        instructions \\
    	Prepare Development MTS Machine & 50\% & coax AMORE into building \\
    	Integrate Nodes into Cluster & 100\% & NAS-0 and SE still must be
                                               included \\
    	Assist Researchers & 100\% & helping out with general problems as they
                                     arise \\
    	\hline
    \end{tabular}
  \end{center}
\end{adjustwidth}



\section{Discussion - Current Milestone}

\subsection{Existing MTS Progress}

\tab The existing MTS continues to suffer from its hardware ailments. While we
were able to get new firmware for the FEC, it did not fix the FEC. Although the
firmware was able to be installed onto the FEC, the FEC's ethernet port suddenly
became inoperable. We do not know if this is a firmware error or a hardware
failure.

\subsection{Development MTS Computer Progress}

\tab We have been provided with the source repository for AMORE! We have cloned
the repository onto the development MTS machine and tried to make it. However,
all documentation on AMORE states that this is not the recommended way of
building AMORE, and so there is very little documentation on the process for
building AMORE via its source. When {\tt make} was run in the root of the
repository, it, of course, ran into some issues. A path variable was incorrectly
configured in one of the internal make files, so we had to overwrite it so that
it pointed to the correct directory. After that was fixed, it complained that it
could not find a particular ROOT file that was not on the machine. Fortunately,
however, the existing MTS has that file, so we copied it over, and it stopped
complaining. Our next hurdle is figuring out how to deal with some type errors
in yet another file.

\subsection{Computing Cluster}

\tab The nodes have been integrated into the cluster! Turns out their boot order
was messed up; the correct order is PXE network boot, CD, then HDD. With PXE
networking booting enabled and set to the highest priority, the nodes will
automatically listen for kickstart files from the CE on boot. This allowed the
CE to send over all the files necessary to install ROCKS 7 and incorporate the
nodes into the cluster! 

\tab This victory is not without its pitfalls, however. A couple of the nodes
were rather uncooperative, and it took us some time to get them sorted
out. Additionally, we are unable to run all the nodes simultaneously due to
issues with the UPSs powering the nodes; if seven nodes are turned on at the
same time, the UPS's breaker is tripped and it shuts off. Until we can solve
this problem, the nodes will be operated on in two groups of ten nodes each,
five for each UPS.

\tab Since the nodes have been brought into the cluster, we began trying to
incorporate some other components; we started with NAS-0. There are {\tt
  insert-ethers} options for NASs, so the process is very similar to that of the
nodes. We modified NAS-0's boot order in the same manner as the nodes, and it
successfully requested its kickstart file from the CE to begin the ROCKS 7
installation. Unfortunately, NAS-0's OS drive was not seen by the installer;
only the data storage drives appeared. 

\subsection{GEM Machines}

\tab The computers in the lab all seem to be suffering a major issue that was
brought into light due to the failure of a hard drive in the Truth PC. There
seemed to be something strange going on, the Truth PC was thought to be running
in RAID 1, which means that with two hard drives if there was a hard drive failure it would not be catastrophic. However upon further investigation there
seemed to be confusion as the BIOS stated that it was "RAID(1)" meaning that it 
was RAID array 1, not RAID 1. The RAID supported on the motherboard of the 
computer can only support RAID 0, which does not conserve storage but instead
expands the two drives as only one recognized drive on the computer. \\

\tab The GEM team seemed to be suffering from some issues with productivity, as
they were running a program that would output a text file showing exactly how
many hits a detector would make based on particle tracks. The problem is the
team needed to gather all the hits of each different type (there are 5 types of
possible hits a particle could do) and averages the amount of hits. They were
doing this manually, due to their lack of knowledge of programming and the
complexity of the text file (which was written very poorly, and quite difficult
to work with) they were working with they figured doing it manually was best. 
Creating a script improved productivity, and provided more accurate results. 

\section{Parts Worked On}

\subsection{Josef Bostik}

\begin{itemize}
\item building AMORE on the MTS Development Machine
\end{itemize}

\subsection{Eric Pereira}

\begin{itemize}
\item building scripts for gem team
\item finding effective backup solutions for GEM computers
\end{itemize}

\subsection{Ryan Wojtyla}

\begin{itemize}
\item integrating cluster components
\item building AMORE on the MTS Development Machine
\end{itemize}

\section{Task Matrix - Next Milestone}

\begin{center}
  \begin{tabular}{|c|c|c|c|}
    \hline
    Task & Josef & Eric & Ryan \\
    \hline
    Polish Cluster Documentation & 10\% & 10\% & 80\% \\
    Polish MTS Documentation & 40\% & 20\% & 40\% \\
    Create MTS Automation Script & 60\% & 10\% & 20\% \\
    Integrate Remainder of Cluster Components & 10\% & 10\% & 80\% \\
    Run Jobs on Cluster & 10\% & 10\% & 80\% \\
    Create GEM conputer Backups & 10\% & 80\% & 10\% \\
    Assist Researchers & 10\% & 80\% & 10\% \\
    \hline
  \end{tabular}
\end{center}

\section{Discussion - Next Milestone}

\subsection{Existing MTS}

\tab The future for the existing MTS looks to be quite grim. Our continuing
inability to repair its vital data collection hardware 

\subsection{Development MTS Computer}
	\tab As it stands we are somewhat stuck with the development MTS. We may be 
	able to find a way to create the makefile for the source of root, or we may 
	need to find another method of building AMORE. As soon as AMORE is built, we 
	will write a script to pass data between AMORE, DATE, and ROOT to create an 
	intuitive usage process for the MTS.
\tab 

\subsection{Computing Cluster}

\tab 

\subsection{GEM Machines}

\tab The GEM Machines need more effective backups, it is known that their RAID
systems are not effective. The plan is to get a large amount of drives and use
a drive as an effective backup, loading the image of the current PC on the
backup. This is a temporary solution, until we are able to utilize more
storage on the cluster, or get RAID cards to use on these PC's. \\
\tab It also seems that there are many more issues with productivity that the
 GEM team is having, although the GEM team does not necessarily view them as
 issues. The group has been conducting many tasks similarly to their previous
 problems, manually looking through text files or running a sequence of programs
 that take quite a while instead of running a script to conduct the same task
 multiple times. In order to increase productivity more scripts need to be
 written that will conduct the more menial tasks. 

\section{Sponsor Feedback}

\subsection{Existing MTS}

\vspace{1in}

\subsection{Development MTS Machine}

\vspace{1in}

\subsection{Computing Cluster}

\vspace{1in}

\subsection{GEM Computers}

\vspace{1in}

\newpage

\subsection{Sponsor Signature}
\vspace{.5in}
 \titledate{Sponsor Signature}
 
\section{Sponsor Evaluation}
\begin{tabular}{|c|c|c|c|c|c|c|c|c|c|c|c|c|c|c|c|c|}
	\hline
	Josef Bostik & 0 & 1 & 2 & 3 & 4 & 5 & 5.5 & 6 & 6.5 & 7 & 7.5 & 8 & 8.5 & 9 & 9.5 & 10\\
	\hline
	Eric Pereira & 0 & 1 & 2 & 3 & 4 & 5 & 5.5 & 6 & 6.5 & 7 & 7.5 & 8 & 8.5 & 9 & 9.5 & 10\\
	\hline
	Ryan Wojtyla & 0 & 1 & 2 & 3 & 4 & 5 & 5.5 & 6 & 6.5 & 7 & 7.5 & 8 & 8.5 & 9 & 9.5 & 10\\
	\hline
\end{tabular}

\end{document}
