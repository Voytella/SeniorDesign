\documentclass[12pt]{article}
\newcommand\tab[1][1cm]{\hspace*{#1}}
\usepackage[utf8]{inputenc}
\usepackage{listings}
\usepackage{hyperref}
\usepackage{color}
\pagenumbering{gobble}
\usepackage{changepage}

\usepackage{makecell}

\definecolor{codegreen}{rgb}{0,0.6,0}
\definecolor{codegray}{rgb}{0.5,0.5,0.5}
\definecolor{codepurple}{rgb}{0.58,0,0.82}
\definecolor{backcolour}{rgb}{0.95,0.95,0.92}

\hypersetup{
	colorlinks,
	citecolor=black,
	filecolor=black,
	linkcolor=black,
	urlcolor=black
}

\lstdefinestyle{mystyle}{
	backgroundcolor=\color{backcolour},   
	commentstyle=\color{codegreen},
	keywordstyle=\color{magenta},
	numberstyle=\tiny\color{codegray},
	stringstyle=\color{codepurple},
	basicstyle=\footnotesize,
	breakatwhitespace=false,         
	breaklines=true,                 
	captionpos=b,                    
	keepspaces=true,                 
	numbers=left,                    
	numbersep=5pt,                  
	showspaces=false,                
	showstringspaces=false,
	showtabs=false,                  
	tabsize=2
}

\lstset{style=mystyle}

\newcommand{\titledate}[2][2.5in]{%
	\noindent%
	\begin{tabular}{@{}p{#1}@{}}
		\\ \hline \\[-.75\normalbaselineskip]
		#2
	\end{tabular} \hspace{1in}
	\begin{tabular}{@{}p{#1}@{}}
		\\ \hline \\[-.75\normalbaselineskip]
		Date
	\end{tabular}
}

% for forcing tables to fit
\usepackage{changepage}

\begin{document}
	

\begin{titlepage}
	
\author{Josef Bostik\\
	Eric Pereira\\
	Ryan Wojtyla\\}
\date{March 18\textsuperscript{th}, 2019}
\title{Revamp of High Energy Physics Laboratory's Computer Systems: Milestone 5}

\maketitle

\end{titlepage}

\tableofcontents

\newpage \pagenumbering{arabic}

\section{High Energy Physics (HEP) Senior Design}

\begin{itemize}
	\item Josef Bostik - \href{mailto:jbostik2015@my.fit.edu}{jbostik2015@my.fit.edu}
	\item Eric Pereira - \href{mailto:epereira2015@my.fit.edu}{epereira2015@my.fit.edu }
	\item Ryan Wojtyla - \href{mailto:rwojtyla2015@my.fit.edu}{rwojtyla2015@my.fit.edu}
\end{itemize}

\section{Faculty Sponsor}

\tab Eraldo Ribeiro - \href{mailto:eribeiro@fit.edu}{eribeiro@fit.edu}

\section{Client}

\tab Marcus Hohlmann - \href{mailto:hohlmann@fit.edu}{hohlmann@fit.edu} \\ 
\tab Head of the Florida Institute of Technology HEP group

\section{Meeting with Faculty Sponsor}

\begin{itemize}
	\item 18 March 2019
\end{itemize}
\section{Meeting with Client}

\begin{itemize}
	\item 11 February 2019
	\item 18 February 2019
	\item 25 February 2019
	\item 11 March 2019
\end{itemize}

\section{Progress of current Milestone}

\begin{adjustwidth}{-2.5cm}{}
  \begin{center}
      \begin{tabular}{|c|c|c|c|}
    	\hline
    	Task & Progress & Notes\\
    	\hline
    	Continue to Care for Existing MTS & 40\% & \\
    	Compile Instructions for MTS Operation & 50\% & improve upon provided
                                                        instructions \\
    	Prepare Development MTS Machine & 50\% & coax AMORE into building \\
    	Integrate Nodes into Cluster & 100\% & NAS-0 and SE still must be
                                               included \\
    	Assist Researchers & 100\% & helping out with general problems as they
                                     arise \\
    	\hline
    \end{tabular}
  \end{center}
\end{adjustwidth}



\section{Discussion - Current Milestone}

\subsection{Existing MTS Progress}

\tab 

\subsection{Development MTS Computer Progress}

\tab We've run into quite a few problems installing AMORE on the development MTS.
	As we discussed last milestone, AMORE is no longer supported by CERN, and
	the repository containing AMORE has been taken down since last semester. As
	such, we contacted somebody who worked in the lab last year, and they turned
	out to have the source code for AMORE. However, all documentation on AMORE states
	that this is not the recommended way of building AMORE, and so there is very
	little documentation on the process for building AMORE via its source. After some
	research, we found out the process for building AMORE via its source and 
	implemented it. however a makefile in /usr/local/etc named Makefile.arch is 
	required, and we have been unable to find the Makefile. After we got 
	stuck here, we tried to sync the version of AMORE that we successfully built 
	(but was unable to run) with our "new" version of root 5.34.38. 
	After exploring all the issues with the old version of AMORE and the "new" 
	version of root, we tried to solve the run  time errors we were having 
	by manually renaming a few of the library so that AMORE
	would recognize them, only to run into some more issues with variable names 
	not being recognized between our "new" version of root and our old version
	of AMORE.

\subsection{Computing Cluster}

\tab 

\subsection{GEM Machines}

\tab 
\section{Parts Worked On}

\subsection{Josef Bostik}

\begin{itemize}
\item Reinstallation of ROOT on MTS Development Machine
\item Building root on MTS Development Machine
\end{itemize}

\subsection{Eric Pereira}

\begin{itemize}
\item 
\end{itemize}

\subsection{Ryan Wojtyla}

\begin{itemize}
\item 
\end{itemize}

\section{Task Matrix - Next Milestone}

\begin{center}

\end{center}

\section{Discussion - Next Milestone}

\subsection{Existing MTS}

\tab 

\subsection{Development MTS Computer}
	As it stands we are somewhat stuck with the development MTS. We may be able to 
	find a way to create the makefile for the source of root, or we may need to 
	find another method of building AMORE. As soon as AMORE is built, we will write
	a script to pass data between AMORE, DATE, and ROOT to create an intuitive usage
	process for the MTS.
\tab 

\subsection{Computing Cluster}

\tab 

\subsection{GEM Machines}

\tab 

\section{Sponsor Feedback}

\subsection{Existing MTS}

\vspace{1in}

\subsection{Development MTS Machine}

\vspace{1in}

\subsection{Computing Cluster}

\vspace{1in}

\subsection{GEM Computers}

\vspace{1in}

\newpage

\subsection{Sponsor Signature}
\vspace{.5in}
 \titledate{Sponsor Signature}
 
\section{Sponsor Evaluation}
\begin{tabular}{|c|c|c|c|c|c|c|c|c|c|c|c|c|c|c|c|c|}
	\hline
	Josef Bostik & 0 & 1 & 2 & 3 & 4 & 5 & 5.5 & 6 & 6.5 & 7 & 7.5 & 8 & 8.5 & 9 & 9.5 & 10\\
	\hline
	Eric Pereira & 0 & 1 & 2 & 3 & 4 & 5 & 5.5 & 6 & 6.5 & 7 & 7.5 & 8 & 8.5 & 9 & 9.5 & 10\\
	\hline
	Ryan Wojtyla & 0 & 1 & 2 & 3 & 4 & 5 & 5.5 & 6 & 6.5 & 7 & 7.5 & 8 & 8.5 & 9 & 9.5 & 10\\
	\hline
\end{tabular}

\end{document}
