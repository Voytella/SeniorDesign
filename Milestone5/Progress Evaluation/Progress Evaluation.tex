\documentclass[12pt]{article}
\newcommand\tab[1][1cm]{\hspace*{#1}}
\usepackage[utf8]{inputenc}
\usepackage{listings}
\usepackage{hyperref}
\usepackage{color}
\pagenumbering{gobble}
\usepackage{changepage}

\usepackage{makecell}

\definecolor{codegreen}{rgb}{0,0.6,0}
\definecolor{codegray}{rgb}{0.5,0.5,0.5}
\definecolor{codepurple}{rgb}{0.58,0,0.82}
\definecolor{backcolour}{rgb}{0.95,0.95,0.92}

\hypersetup{
	colorlinks,
	citecolor=black,
	filecolor=black,
	linkcolor=black,
	urlcolor=black
}

\lstdefinestyle{mystyle}{
	backgroundcolor=\color{backcolour},   
	commentstyle=\color{codegreen},
	keywordstyle=\color{magenta},
	numberstyle=\tiny\color{codegray},
	stringstyle=\color{codepurple},
	basicstyle=\footnotesize,
	breakatwhitespace=false,         
	breaklines=true,                 
	captionpos=b,                    
	keepspaces=true,                 
	numbers=left,                    
	numbersep=5pt,                  
	showspaces=false,                
	showstringspaces=false,
	showtabs=false,                  
	tabsize=2
}

\lstset{style=mystyle}

\newcommand{\titledate}[2][2.5in]{%
	\noindent%
	\begin{tabular}{@{}p{#1}@{}}
		\\ \hline \\[-.75\normalbaselineskip]
		#2
	\end{tabular} \hspace{1in}
	\begin{tabular}{@{}p{#1}@{}}
		\\ \hline \\[-.75\normalbaselineskip]
		Date
	\end{tabular}
}

% for forcing tables to fit
\usepackage{changepage}

\begin{document}
	

\begin{titlepage}
	
\author{Josef Bostik\\
	Eric Pereira\\
	Ryan Wojtyla\\}
\date{February 11\textsuperscript{th}, 2019}
\title{Revamp of High Energy Physics Laboratory's Computer Systems: Milestone 4}

\maketitle

\end{titlepage}

\tableofcontents

\newpage \pagenumbering{arabic}

\section{High Energy Physics (HEP) Senior Design}

\begin{itemize}
	\item Josef Bostik - \href{mailto:jbostik2015@my.fit.edu}{jbostik2015@my.fit.edu}
	\item Eric Pereira - \href{mailto:epereira2015@my.fit.edu}{epereira2015@my.fit.edu }
	\item Ryan Wojtyla - \href{mailto:rwojtyla2015@my.fit.edu}{rwojtyla2015@my.fit.edu}
\end{itemize}

\section{Faculty Sponsor}

\tab Eraldo Ribeiro - \href{mailto:eribeiro@fit.edu}{eribeiro@fit.edu}

\section{Client}

\tab Marcus Hohlmann - \href{mailto:hohlmann@fit.edu}{hohlmann@fit.edu} \\ 
\tab Head of the Florida Institute of Technology HEP group

\section{Meeting with Faculty Sponsor}

\begin{itemize}
	\item 11 February 2019
\end{itemize}
\section{Meeting with Client}

\begin{itemize}
\item 07 January 2019
\item 14 January 2019
\item 28 January 2019
\item 04 February 2019
\end{itemize}

\section{Progress of current Milestone}

\begin{adjustwidth}{-3.5cm}{}
  \begin{center}
    \begin{tabular}{|c|c|c|c|c|c|}
      \hline
      Task & \% Completion & Eric & Josef & Ryan & To Do \\
      \hline
      Integrate cluster components. & 0\% & 0\% & 0\% & 0\% & find new solutions \\
      Rewire the existing MTS. & 90\% & 70\% & 20\% & 0\% & miscellaneous touch ups \\
      Resolve software issues with development MTS. & 60\% & 0\% & 40\% & 20\% &
                                                                                 reinstall
                                                                                 AMORE
      \\
      \hline   
    \end{tabular}
  \end{center}
\end{adjustwidth}

\section{Discussion - Current Milestone}

\subsection{Existing MTS Progress}

\tab Although the FEC firmware has been slow coming, we have been able to improve the
MTS. Its cabling is quite messy, which makes rearranging components and
diagnosing issues fairly cumbersome. We have exchanged some uncharacteristically
long cables for more appropriate shorter ones, and generally cleaned up the
other cables to be more presentable.

\subsection{Development MTS Computer Progress}

\tab In an attempt to deal with the ROOT and AMORE issues plaguing the
development machine, we have opted to reinstall both packages. We replaced the
installed ROOT with a slightly newer version, which has solved the issues
surrounding ROOT. After attempting to get AMORE to work with the new version of
ROOT, we eventually decided to reinstall AMORE as well. Unfortunately, however,
the repository on which AMORE is hosted is no longer available. The repository
maintainers, though, have agreed to send us the packages we need.

\tab We are working with the researchers working on the MTS to put together a
document detailing the operation of the existing system. Once we have a
through understanding of the MTS's system, we can begin work on the software
wrapper for the development MTS.

\subsection{Computing Cluster}

\tab The integration of the nodes into the cluster has been proving itself to be
very challenging. The process as it is detailed in the ROCKS installation manual
begins by booting the node with the ROCKS Kernel CD used to created the head
node. A tool on the head node, {\tt insert-ethers}, then listens
for a DHCP request sent by the node when it starts up. This catalogs the node in
the head node's database. The node is then supposed to request a kickstart file
from the head node in order to be installed with all the appropriate software
and configurations. Unfortunately, we are having a rather difficult time coaxing
that request from the node. The ROCKS Kernel CD seems only interested in booting
the node as a new head node without requesting a kickstart file.

\tab A ray of hope had shone, however, when, while performing the unrelated task
of installing CERN CentOS 7 onto a machine, we discovered that if a certain URL
is provided in the ``Installation Source'' section of the Anaconda installer,
shared by the ROCKS Kernel CD and the CERN CentOS CD, a long list of software
packages is made available to customize what kind of system we would like to
install, including a ``Compute Node'' package. Enlivened by this discovery, we
immediately tried the URL in the ROCKS Kernel CD on the node, but no such list
revealed itself. We then booted the node into the CERN CentOS 7 CD and tried the
URL. The list appeared! We selected ``Compute Node'' and began the
installation. The node, however, never requested a kickstart file from the head
node. 

\subsection{GEM Machines}

\tab There was a major failure in the Truth PC, this PC is an integral part of
the GEM machine systems as it is one of the few PC's with ROOT and AMORE 
properly installed. However, getting other systems to run RAID has proven well, as most machines already have 2 drives installed that are the same size, they
just were not adjusted to run in RAID 1 in the bios in OS, so a few tweaks 
allows them to work. \\

\tab It seems as though there are other small issues that are impacting the
usability of the lab right now, one of them being that the Beauty PC has some
hardware troubles. Upon the installation of a new OS on this PC 2 of its 4 
monitors refuse to work properly for some reason. After some testing it seems
to be an issue with the drivers, and the proper drivers have been downloaded but
not yet installed as the computer needs to reset for proper driver installation 
and we have been unable to reset it. After a reset the Beauty PC should provide
a more productive experience to the GEM team.  

\section{Parts Worked On}

\subsection{Josef Bostik}

\begin{itemize}
\item reinstallation of ROOT on the MTS Development Machine
\item reinstallation of AMORE on the MTS Development Machine
\end{itemize}

\subsection{Eric Pereira}

\begin{itemize}
\item Recabling the existing MTS
\item Exploring backup solutions
\item Saving data from Truth PC. 
\item Beauty PC repair. 
\end{itemize}

\subsection{Ryan Wojtyla}

\begin{itemize}
\item integrating compute nodes into the cluster
\item reinstallation of AMORE on the MTS Development Machine
\end{itemize}

\section{Task Matrix - Next Milestone}

\begin{center}
  \begin{tabular}{|c|c|c|c|}
    \hline
    Task & Eric & Josef & Ryan \\
    \hline
    Continue to Care for Existing MTS & 60\% & 20\% & 20\% \\
    Compile Instructions for MTS Operation & 20\% & 40\% & 40\% \\
    Prepare Development MTS Machine & 20\% & 50\% & 30\% \\
    Integrate Nodes into Cluster & 10\% & 10\% & 80\% \\
    Assist Researchers & 70\% & 10\% & 20\% \\
    \hline
  \end{tabular}
\end{center}

\section{Discussion - Next Milestone}

\subsection{Existing MTS}

\tab While we wait for the FEC firmware to be made available to us, we can
continue to clean up the physical detector. We will also be creating a manual
in collaboration with the MTS's researchers detailing the operation of its
software.

\subsection{Development MTS Computer}

\tab Once the new AMORE package arrives, we will resume the process of
completing the construction of the development MTS's software environment. We
will also be able to begin construction of the wrapper once key parts of the
manual have been completed.

\subsection{Computing Cluster}

\tab Further efforts will be made to integrate the nodes into the cluster. Once
the nodes are integrated, the storage element (SE) and storage units (NASs) must
also be integrated.

\subsection{GEM Machines}

\tab The Truth PC has been an example to the GEM team as to what can occur if
data is not backed up properly. It seems that the Truth PC, one of the few designated PC's with AMORE and ROOT properly installed, and it has suffered a
RAID issue, and seems to have a complete drive failure preventing the computer
from working properly. Although The GEM team has lost one machine it can be 
prevented from doing so in the future, so all machines will have backup disks
and run RAID 1 systems until the cluster is back up online, once the cluster is
back up online the systems can be backed up on the computer clusters storage. 

\section{Sponsor Feedback}

\subsection{Existing MTS}

\vspace{1in}

\subsection{Development MTS Machine}

\vspace{1in}

\subsection{Computing Cluster}

\vspace{1in}

\subsection{GEM Computers}

\vspace{1in}

\newpage

\subsection{Sponsor Signature}
\vspace{.5in}
 \titledate{Sponsor Signature}
 
\section{Sponsor Evaluation}
\begin{tabular}{|c|c|c|c|c|c|c|c|c|c|c|c|c|c|c|c|c|}
	\hline
	Josef Bostik & 0 & 1 & 2 & 3 & 4 & 5 & 5.5 & 6 & 6.5 & 7 & 7.5 & 8 & 8.5 & 9 & 9.5 & 10\\
	\hline
	Eric Pereira & 0 & 1 & 2 & 3 & 4 & 5 & 5.5 & 6 & 6.5 & 7 & 7.5 & 8 & 8.5 & 9 & 9.5 & 10\\
	\hline
	Ryan Wojtyla & 0 & 1 & 2 & 3 & 4 & 5 & 5.5 & 6 & 6.5 & 7 & 7.5 & 8 & 8.5 & 9 & 9.5 & 10\\
	\hline
\end{tabular}

\end{document}
