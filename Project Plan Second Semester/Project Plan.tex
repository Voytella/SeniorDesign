\documentclass[12pt]{article}
\newcommand\tab[1][1cm]{\hspace*{#1}}
\usepackage[utf8]{inputenc}
\usepackage{listings}
\usepackage{hyperref}
\usepackage{multirow}
\usepackage{color}
\usepackage{graphicx}
\pagenumbering{gobble}


\newcommand{\doubleSignature}[2]{
	\begin{center}
		
	\end{center}
	\vspace{2cm}
	
	\noindent
	\begin{tabular}{lcl}
		\rule{7cm}{1pt} & \hspace{2cm} & \rule{3cm}{1pt} \\
		#1 & & #2
	\end{tabular}
	\vspace{1cm}
}

\definecolor{codegreen}{rgb}{0,0.6,0}
\definecolor{codegray}{rgb}{0.5,0.5,0.5}
\definecolor{codepurple}{rgb}{0.58,0,0.82}
\definecolor{backcolour}{rgb}{0.95,0.95,0.92}

\hypersetup{
	colorlinks,
	citecolor=black,
	filecolor=black,
	linkcolor=black,
	urlcolor=black
}

\lstdefinestyle{mystyle}{
	backgroundcolor=\color{backcolour},   
	commentstyle=\color{codegreen},
	keywordstyle=\color{magenta},
	numberstyle=\tiny\color{codegray},
	stringstyle=\color{codepurple},
	basicstyle=\footnotesize,
	breakatwhitespace=false,         
	breaklines=true,                 
	captionpos=b,                    
	keepspaces=true,                 
	numbers=left,                    
	numbersep=5pt,                  
	showspaces=false,                
	showstringspaces=false,
	showtabs=false,                  
	tabsize=2
}

\lstset{style=mystyle}

\begin{document}
\begin{titlepage}
	

\author{Josef Bostik\\
	Eric Pereira\\
	Ryan Wojtlya\\}
\date{January 14\textsuperscript{th}, 2019}
\title{Project Plan}
\maketitle
\end{titlepage}
\tableofcontents
\newpage
\pagenumbering{arabic}

\section{Project Title}
\tab Upgrade and Update of Computer Systems within Dr. Hohlmann's High Energy Physics (HEP) Research Groups
\section{Names and Email Addresses of Team Members}
\tab
\begin{tabular}{| c | c |}
	\hline
	Joseph Bostik & jbostik2015@my.fit.edu \\
	\hline
	Eric Pereira & epereira2015@my.fit.edu \\
	\hline
	Ryan Wojtyla & rwojtyla2015@my.fit.edu \\
	\hline
\end{tabular}

\section{Faculty Sponsor}
Dr. Eraldo Ribeiro, eribeiro@fit.edu

\section{Client}
Dr. Marcus Hohlmann, mhohlmann@fit.edu

\section{Meeting(s) with the client for developing this plan}
\tab Weekly Monday meetings

\section{Goal and Motivation}

\section{Approach}
\subsection{Compute Cluster}
Before any of the necessary software can be installed onto the cluster, the
cluster's components must be connected. The first challenge is to use the tools
provided by ROCKS to integrate the nodes, the SE, and the NAS into the cluster,
commanded by the CE, which already has ROCKS 7 installed. Once the cluster is
reconstructed, the job manager, HTCondor, can be installed and configured. With
the ability to receive, route, and process jobs, all that will remain is
bringing the cluster back into operation with the Open Science Grid through the
installation and configuration of required OSG software.
\subsection{Computer Systems}
\subsection{Muon Tomography Station (MTS)}

\section{Novel Features/Functionalities}
\subsection{Compute Cluster}
With the cluster updated, the research group will regain the ability to run
compute jobs locally, thus increasing their capacity for processing
data. Additionally, scripts will be created that will ease the researchers'
interaction with HTCondor so that submitting jobs will be less painful.
\subsection{Computer Systems}
\subsection{MTS}

\section{Technical Challenges}
Technical challenges for the cluster, while unknown at present, will
undoubtedly manifest themselves as miscellaneous problems along the course of
component integration and software configuration. 
\section{Design}

\section{Progress Summary}

\section{Milestone 4 (Feb 11)}
\begin{itemize}
  \item Fully integrate all parts of the cluster.
\end{itemize}

\section{Milestone 5 (Mar 18)}
\begin{itemize}
  \item Install and configure HTCondor.
  \item Create job submission script.
\end{itemize}

\section{Milestone 6 (April 15)}
\begin{itemize}
  \item Integrate the cluster into OSG's compute grid.
\end{itemize}

\section{Task Matrix for Milestone 4}

\section{Description of each planned task for Milestone 4}

\subsection{Cluster Integration}
This task involves fully incorporating each individual cluster component into
the cluster as a whole. ROCKS 7 must first be installed onto each node, the SE,
and NAS-0 to allow for the tools on the CE to access each component and
configure it to be part of the cluster.

\section{Approval from Faculty Sponsor}
\paragraph{\tab "I have discussed with the team and approve this project plan. I will evaluate the progress and assign a grade fo reach of the three milestones"}
\doubleSignature{Signature}{Date}

\end{document}
