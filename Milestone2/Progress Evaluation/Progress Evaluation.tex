\documentclass[12pt]{article}
\newcommand\tab[1][1cm]{\hspace*{#1}}
\usepackage[utf8]{inputenc}
\usepackage{listings}
\usepackage{hyperref}
\usepackage{color}
\pagenumbering{gobble}
\usepackage{changepage}

\usepackage{makecell}

\definecolor{codegreen}{rgb}{0,0.6,0}
\definecolor{codegray}{rgb}{0.5,0.5,0.5}
\definecolor{codepurple}{rgb}{0.58,0,0.82}
\definecolor{backcolour}{rgb}{0.95,0.95,0.92}

\hypersetup{
	colorlinks,
	citecolor=black,
	filecolor=black,
	linkcolor=black,
	urlcolor=black
}

\lstdefinestyle{mystyle}{
	backgroundcolor=\color{backcolour},   
	commentstyle=\color{codegreen},
	keywordstyle=\color{magenta},
	numberstyle=\tiny\color{codegray},
	stringstyle=\color{codepurple},
	basicstyle=\footnotesize,
	breakatwhitespace=false,         
	breaklines=true,                 
	captionpos=b,                    
	keepspaces=true,                 
	numbers=left,                    
	numbersep=5pt,                  
	showspaces=false,                
	showstringspaces=false,
	showtabs=false,                  
	tabsize=2
}

\lstset{style=mystyle}

\begin{document}

\begin{titlepage}
	
\author{Josef Bostik\\
	Eric Pereira\\
	Ryan Wojtyla\\}
\date{October 29\textsuperscript{th}, 2018}
\title{Progress Evaluation 2}

\maketitle

\end{titlepage}

\tableofcontents

\newpage \pagenumbering{arabic}

\section{HEP Senior Design}

\begin{itemize}
	\item Josef Bostik - \href{mailto:jbostik2015@my.fit.edu}{jbostik2015@my.fit.edu}
	\item Eric Pereira - \href{mailto:epereira2015@my.fit.edu}{epereira2015@my.fit.edu }
	\item Ryan Wojtyla - \href{mailto:rwojtyla2015@my.fit.edu}{rwojtyla2015@my.fit.edu}
\end{itemize}

\section{Faculty Sponsor}

\tab Eraldo Ribeiro - \href{mailto:eribeiro@fit.edu}{eribeiro@fit.edu}

\section{Client}

\tab Marcus Hohlmann - \href{mailto:hohlmann@fit.edu}{hohlmann@fit.edu} \\ 
\tab Head of the Florida Tech HEP group

\section{Meeting with Faculty Sponsor}
\tab 29 October 2018

\section{Meeting with Client}

\begin{itemize}
\item[] 01 October 2018
\item[] 08 October 2018
\item[] 15 October 2018
\item[] 22 October 2018
\end{itemize}

\section{Progress of current Milestone}

  \begin{center}
  \begin{tabular}{|c|c|c|c|c|c|}
    \hline
    Task & \% Completion & Ryan & Eric & Josef & To Do \\
    \hline
    Repair MTS Data Collection & 100\% & 40\% & 40\% & 20\% & none \\
    Install DATE and AMORE & 100\% & 40\% & 20\% & 40\% & none \\
    Install ROCKS & 50\% & 30\% & 10\% & 10\% & boot into ROCKS \\
    Testing Hard Drives & 100\% & 30\% & 50\% & 20\% & none \\
    \hline   
  \end{tabular}
\end{center}

\section{Discussion - Current Milestone}

\subsection{Existing MTS Progress}

\tab At the beginning of this Milestone, the MTS was unable to collect data due
to software issues. We were able to locate the faulty script and edit it
accordingly to eliminate the issue. This, however, brought to light a hardware
problem: one of the Front End Controllers (FECs) was not transmitting any
data. After some troubleshooting, it was determined that the FEC was working
properly, but its Analogue to Digital Converter (ADC) was not. Fortunately, we
had an extra ADC, so we replaced the faulty component. The MTS is now capable of
taking data! Data collection is not entirely fixed, however. There are some
software issues regarding how the data is processed and some hardware issues
regarding the data collection detectors.

\subsection{Development MTS Computer Progress}

\tab Our first challenge was to find the repository from which we could obtain
the software to be installed, DATE and AMORE. While AMORE was present in the
CERN CentOS 7 repository we had access to, DATE was not. The latest version of
DATE is only found in the CERN Scientific Linux 6 repository. Since we did not
have that repository, we manually created the repository file using the CERN
CentOS 7 file as a blueprint; we changed all instances of {\tt cc7}
(\textbf{C}ERN \textbf{C}entOS \textbf{7}) to {\tt slc6} (\textbf{S}cientific
\textbf{L}inux \textbf{C}ERN \textbf{6}). Miraculously, we were able to access
the repository and install the software we needed! The current challenge is
installing the drivers for the MTS's hardware.

\subsection{Computing Cluster}

\tab The cluster continues to be an incredible challenge. Since the Anaconda
installer does not recognize the internet while on the cluster, we figured it
might on another machine. As long as ROCKS is installed properly onto a drive,
we can copy everything over to the cluster's drives. We loaded the installer
onto a separate machine and attempted to install ROCKS onto a flash drive. While
Anaconda recognized the internet and began the installation process smoothly, it
failed partway through. It does not seem this method will work.

\tab For the sake of investigation, I loaded a Ubuntu LiveCD onto the cluster to
examine the contents of the drives, and what I found surprised me. There are
CentOS 7 images loaded onto the drives already in the cluster! Perhaps the
original installation was not completely botched. The drives are having a very
hard time booting, however, so the next step is to investigate GRUB, the boot
loader.

\subsection{GEM Machines}

\tab The GEM Machines are behaving well. The current XRay-PC issue is not much of an issue, we were not able to set up the initial PC with the XRay drivers, however there is another computer that runs the XRay fine so it is not an issue. We also tested some hard drives for the machines, with the intent that writing 0's may repair some damaged hard drives. Sadly this did not work, however there are still a few more hard drives to test. 

\section{Parts Worked On}

\subsection{Josef Bostik}

\begin{itemize}
  \item software installation on development MTS computer
  \item software configuration on development MTS computer
  \item software repair of existing MTS computer
\end{itemize}

\subsection{Eric Pereira}

\begin{itemize}
  \item test hard drives for the compute cluster and GEM machines
  \item software repair of existing MTS computer
  \item hardware repair of existing MTS computer
\end{itemize}

\subsection{Ryan Wojtyla}

\begin{itemize}
  \item troubleshooting the computing cluster
  \item hardware repair of existing MTS computer
  \item software installation on development MTS computer
\end{itemize}

\section{Task Matrix - Next Milestone}

\begin{center}
  \begin{tabular}{|c|c|c|c|}
    \hline
    Task & Ryan & Eric & Josef \\
    \hline
    Repair the existing MTS. & 30\% & 40\% & 30\% \\
    Install hardware drivers onto the development MTS machine. & 40\% & 40\% &
                                                                               20\%
    \\
    Boot the cluster into ROCKS. & 80\% & 10\% & 10\% \\
    Continue setting up the GEM team's machines. & 15\% & 70\% & 15\% \\
    \hline
  \end{tabular}
\end{center}

\section{Discussion - Next Milestone}

\subsection{Existing MTS}

\tab Although the MTS is once again (mostly) capable of collecting data, it is
having trouble processing it. A script is run on the raw data file that
transforms the data into a form usable by data visualization tools, namely
ROOT. The processed data, however, is not being stored in the expected location,
or any location for that matter. The current challenge is to diagnose the issue
and remedy the situation.

\subsection{Development MTS Computer}

\tab With the required software installed, the next step is to prepare for the
attachment of MTS hardware. The installation of the drivers for that hardware,
however, is proving to be quite an obstacle. Throughout the software
installation, we have been following a guide put together by a researcher
setting up a system similar to our own. While he experienced issues while
installing his drivers, ours are different. The current challenge is to figure
out what is wrong with the system's configuration that is preventing the drivers
from being installed.

\subsection{Computing Cluster}

\tab With the discovery of CentOS 7 images on the cluster's drives, the question
has changed from ``How do we install ROCKS onto the cluster?'' to ``How do we
get the cluster to boot into ROCKS?''. We have been conducting extensive
research into the operation of the boot loader, GRUB, so that we can better
understand how a working system is constructed in order to more effectively
diagnose booting issues on the cluster. The current challenge is to load ROCKS
from the cluster's drives.

\subsection{GEM Machines}

\tab The GEM computers are optimal for the most part with very minor issues. All the parts are working, resources have been allocated successfully, drivers are optimized and working. Currently the largest issue is with one of the computers (named Charm) has issues connecting to a keyboard on occasion. This is a strange bug, and resetting usually fixes the problem, however it, on the occasion, will have this error. It is unknown whether this is a hardware or software issue, will be investigated. There are also some hard drives that still need to be tested, they are currently not in use and sitting in a cabinet, and it is unknown if they work or not, they need to be tested and have 0's written on them. 

\section{Sponsor Feedback}

\subsection{Existing MTS}

\vspace{1in}

\subsection{Development MTS Machine}

\vspace{1in}

\subsection{Computing Cluster}

\vspace{1in}

\subsection{GEM Computers}

\vspace{1in}

\end{document}
